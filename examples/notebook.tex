
% Default to the notebook output style

    


% Inherit from the specified cell style.




    
\documentclass[11pt]{article}

    
    
    \usepackage[T1]{fontenc}
    % Nicer default font (+ math font) than Computer Modern for most use cases
    \usepackage{mathpazo}

    % Basic figure setup, for now with no caption control since it's done
    % automatically by Pandoc (which extracts ![](path) syntax from Markdown).
    \usepackage{graphicx}
    % We will generate all images so they have a width \maxwidth. This means
    % that they will get their normal width if they fit onto the page, but
    % are scaled down if they would overflow the margins.
    \makeatletter
    \def\maxwidth{\ifdim\Gin@nat@width>\linewidth\linewidth
    \else\Gin@nat@width\fi}
    \makeatother
    \let\Oldincludegraphics\includegraphics
    % Set max figure width to be 80% of text width, for now hardcoded.
    \renewcommand{\includegraphics}[1]{\Oldincludegraphics[width=.8\maxwidth]{#1}}
    % Ensure that by default, figures have no caption (until we provide a
    % proper Figure object with a Caption API and a way to capture that
    % in the conversion process - todo).
    \usepackage{caption}
    \DeclareCaptionLabelFormat{nolabel}{}
    \captionsetup{labelformat=nolabel}

    \usepackage{adjustbox} % Used to constrain images to a maximum size 
    \usepackage{xcolor} % Allow colors to be defined
    \usepackage{enumerate} % Needed for markdown enumerations to work
    \usepackage{geometry} % Used to adjust the document margins
    \usepackage{amsmath} % Equations
    \usepackage{amssymb} % Equations
    \usepackage{textcomp} % defines textquotesingle
    % Hack from http://tex.stackexchange.com/a/47451/13684:
    \AtBeginDocument{%
        \def\PYZsq{\textquotesingle}% Upright quotes in Pygmentized code
    }
    \usepackage{upquote} % Upright quotes for verbatim code
    \usepackage{eurosym} % defines \euro
    \usepackage[mathletters]{ucs} % Extended unicode (utf-8) support
    \usepackage[utf8x]{inputenc} % Allow utf-8 characters in the tex document
    \usepackage{fancyvrb} % verbatim replacement that allows latex
    \usepackage{grffile} % extends the file name processing of package graphics 
                         % to support a larger range 
    % The hyperref package gives us a pdf with properly built
    % internal navigation ('pdf bookmarks' for the table of contents,
    % internal cross-reference links, web links for URLs, etc.)
    \usepackage{hyperref}
    \usepackage{longtable} % longtable support required by pandoc >1.10
    \usepackage{booktabs}  % table support for pandoc > 1.12.2
    \usepackage[inline]{enumitem} % IRkernel/repr support (it uses the enumerate* environment)
    \usepackage[normalem]{ulem} % ulem is needed to support strikethroughs (\sout)
                                % normalem makes italics be italics, not underlines
    

    
    
    % Colors for the hyperref package
    \definecolor{urlcolor}{rgb}{0,.145,.698}
    \definecolor{linkcolor}{rgb}{.71,0.21,0.01}
    \definecolor{citecolor}{rgb}{.12,.54,.11}

    % ANSI colors
    \definecolor{ansi-black}{HTML}{3E424D}
    \definecolor{ansi-black-intense}{HTML}{282C36}
    \definecolor{ansi-red}{HTML}{E75C58}
    \definecolor{ansi-red-intense}{HTML}{B22B31}
    \definecolor{ansi-green}{HTML}{00A250}
    \definecolor{ansi-green-intense}{HTML}{007427}
    \definecolor{ansi-yellow}{HTML}{DDB62B}
    \definecolor{ansi-yellow-intense}{HTML}{B27D12}
    \definecolor{ansi-blue}{HTML}{208FFB}
    \definecolor{ansi-blue-intense}{HTML}{0065CA}
    \definecolor{ansi-magenta}{HTML}{D160C4}
    \definecolor{ansi-magenta-intense}{HTML}{A03196}
    \definecolor{ansi-cyan}{HTML}{60C6C8}
    \definecolor{ansi-cyan-intense}{HTML}{258F8F}
    \definecolor{ansi-white}{HTML}{C5C1B4}
    \definecolor{ansi-white-intense}{HTML}{A1A6B2}

    % commands and environments needed by pandoc snippets
    % extracted from the output of `pandoc -s`
    \providecommand{\tightlist}{%
      \setlength{\itemsep}{0pt}\setlength{\parskip}{0pt}}
    \DefineVerbatimEnvironment{Highlighting}{Verbatim}{commandchars=\\\{\}}
    % Add ',fontsize=\small' for more characters per line
    \newenvironment{Shaded}{}{}
    \newcommand{\KeywordTok}[1]{\textcolor[rgb]{0.00,0.44,0.13}{\textbf{{#1}}}}
    \newcommand{\DataTypeTok}[1]{\textcolor[rgb]{0.56,0.13,0.00}{{#1}}}
    \newcommand{\DecValTok}[1]{\textcolor[rgb]{0.25,0.63,0.44}{{#1}}}
    \newcommand{\BaseNTok}[1]{\textcolor[rgb]{0.25,0.63,0.44}{{#1}}}
    \newcommand{\FloatTok}[1]{\textcolor[rgb]{0.25,0.63,0.44}{{#1}}}
    \newcommand{\CharTok}[1]{\textcolor[rgb]{0.25,0.44,0.63}{{#1}}}
    \newcommand{\StringTok}[1]{\textcolor[rgb]{0.25,0.44,0.63}{{#1}}}
    \newcommand{\CommentTok}[1]{\textcolor[rgb]{0.38,0.63,0.69}{\textit{{#1}}}}
    \newcommand{\OtherTok}[1]{\textcolor[rgb]{0.00,0.44,0.13}{{#1}}}
    \newcommand{\AlertTok}[1]{\textcolor[rgb]{1.00,0.00,0.00}{\textbf{{#1}}}}
    \newcommand{\FunctionTok}[1]{\textcolor[rgb]{0.02,0.16,0.49}{{#1}}}
    \newcommand{\RegionMarkerTok}[1]{{#1}}
    \newcommand{\ErrorTok}[1]{\textcolor[rgb]{1.00,0.00,0.00}{\textbf{{#1}}}}
    \newcommand{\NormalTok}[1]{{#1}}
    
    % Additional commands for more recent versions of Pandoc
    \newcommand{\ConstantTok}[1]{\textcolor[rgb]{0.53,0.00,0.00}{{#1}}}
    \newcommand{\SpecialCharTok}[1]{\textcolor[rgb]{0.25,0.44,0.63}{{#1}}}
    \newcommand{\VerbatimStringTok}[1]{\textcolor[rgb]{0.25,0.44,0.63}{{#1}}}
    \newcommand{\SpecialStringTok}[1]{\textcolor[rgb]{0.73,0.40,0.53}{{#1}}}
    \newcommand{\ImportTok}[1]{{#1}}
    \newcommand{\DocumentationTok}[1]{\textcolor[rgb]{0.73,0.13,0.13}{\textit{{#1}}}}
    \newcommand{\AnnotationTok}[1]{\textcolor[rgb]{0.38,0.63,0.69}{\textbf{\textit{{#1}}}}}
    \newcommand{\CommentVarTok}[1]{\textcolor[rgb]{0.38,0.63,0.69}{\textbf{\textit{{#1}}}}}
    \newcommand{\VariableTok}[1]{\textcolor[rgb]{0.10,0.09,0.49}{{#1}}}
    \newcommand{\ControlFlowTok}[1]{\textcolor[rgb]{0.00,0.44,0.13}{\textbf{{#1}}}}
    \newcommand{\OperatorTok}[1]{\textcolor[rgb]{0.40,0.40,0.40}{{#1}}}
    \newcommand{\BuiltInTok}[1]{{#1}}
    \newcommand{\ExtensionTok}[1]{{#1}}
    \newcommand{\PreprocessorTok}[1]{\textcolor[rgb]{0.74,0.48,0.00}{{#1}}}
    \newcommand{\AttributeTok}[1]{\textcolor[rgb]{0.49,0.56,0.16}{{#1}}}
    \newcommand{\InformationTok}[1]{\textcolor[rgb]{0.38,0.63,0.69}{\textbf{\textit{{#1}}}}}
    \newcommand{\WarningTok}[1]{\textcolor[rgb]{0.38,0.63,0.69}{\textbf{\textit{{#1}}}}}
    
    
    % Define a nice break command that doesn't care if a line doesn't already
    % exist.
    \def\br{\hspace*{\fill} \\* }
    % Math Jax compatability definitions
    \def\gt{>}
    \def\lt{<}
    % Document parameters
    \title{homework4}
    
    
    

    % Pygments definitions
    
\makeatletter
\def\PY@reset{\let\PY@it=\relax \let\PY@bf=\relax%
    \let\PY@ul=\relax \let\PY@tc=\relax%
    \let\PY@bc=\relax \let\PY@ff=\relax}
\def\PY@tok#1{\csname PY@tok@#1\endcsname}
\def\PY@toks#1+{\ifx\relax#1\empty\else%
    \PY@tok{#1}\expandafter\PY@toks\fi}
\def\PY@do#1{\PY@bc{\PY@tc{\PY@ul{%
    \PY@it{\PY@bf{\PY@ff{#1}}}}}}}
\def\PY#1#2{\PY@reset\PY@toks#1+\relax+\PY@do{#2}}

\expandafter\def\csname PY@tok@w\endcsname{\def\PY@tc##1{\textcolor[rgb]{0.73,0.73,0.73}{##1}}}
\expandafter\def\csname PY@tok@c\endcsname{\let\PY@it=\textit\def\PY@tc##1{\textcolor[rgb]{0.25,0.50,0.50}{##1}}}
\expandafter\def\csname PY@tok@cp\endcsname{\def\PY@tc##1{\textcolor[rgb]{0.74,0.48,0.00}{##1}}}
\expandafter\def\csname PY@tok@k\endcsname{\let\PY@bf=\textbf\def\PY@tc##1{\textcolor[rgb]{0.00,0.50,0.00}{##1}}}
\expandafter\def\csname PY@tok@kp\endcsname{\def\PY@tc##1{\textcolor[rgb]{0.00,0.50,0.00}{##1}}}
\expandafter\def\csname PY@tok@kt\endcsname{\def\PY@tc##1{\textcolor[rgb]{0.69,0.00,0.25}{##1}}}
\expandafter\def\csname PY@tok@o\endcsname{\def\PY@tc##1{\textcolor[rgb]{0.40,0.40,0.40}{##1}}}
\expandafter\def\csname PY@tok@ow\endcsname{\let\PY@bf=\textbf\def\PY@tc##1{\textcolor[rgb]{0.67,0.13,1.00}{##1}}}
\expandafter\def\csname PY@tok@nb\endcsname{\def\PY@tc##1{\textcolor[rgb]{0.00,0.50,0.00}{##1}}}
\expandafter\def\csname PY@tok@nf\endcsname{\def\PY@tc##1{\textcolor[rgb]{0.00,0.00,1.00}{##1}}}
\expandafter\def\csname PY@tok@nc\endcsname{\let\PY@bf=\textbf\def\PY@tc##1{\textcolor[rgb]{0.00,0.00,1.00}{##1}}}
\expandafter\def\csname PY@tok@nn\endcsname{\let\PY@bf=\textbf\def\PY@tc##1{\textcolor[rgb]{0.00,0.00,1.00}{##1}}}
\expandafter\def\csname PY@tok@ne\endcsname{\let\PY@bf=\textbf\def\PY@tc##1{\textcolor[rgb]{0.82,0.25,0.23}{##1}}}
\expandafter\def\csname PY@tok@nv\endcsname{\def\PY@tc##1{\textcolor[rgb]{0.10,0.09,0.49}{##1}}}
\expandafter\def\csname PY@tok@no\endcsname{\def\PY@tc##1{\textcolor[rgb]{0.53,0.00,0.00}{##1}}}
\expandafter\def\csname PY@tok@nl\endcsname{\def\PY@tc##1{\textcolor[rgb]{0.63,0.63,0.00}{##1}}}
\expandafter\def\csname PY@tok@ni\endcsname{\let\PY@bf=\textbf\def\PY@tc##1{\textcolor[rgb]{0.60,0.60,0.60}{##1}}}
\expandafter\def\csname PY@tok@na\endcsname{\def\PY@tc##1{\textcolor[rgb]{0.49,0.56,0.16}{##1}}}
\expandafter\def\csname PY@tok@nt\endcsname{\let\PY@bf=\textbf\def\PY@tc##1{\textcolor[rgb]{0.00,0.50,0.00}{##1}}}
\expandafter\def\csname PY@tok@nd\endcsname{\def\PY@tc##1{\textcolor[rgb]{0.67,0.13,1.00}{##1}}}
\expandafter\def\csname PY@tok@s\endcsname{\def\PY@tc##1{\textcolor[rgb]{0.73,0.13,0.13}{##1}}}
\expandafter\def\csname PY@tok@sd\endcsname{\let\PY@it=\textit\def\PY@tc##1{\textcolor[rgb]{0.73,0.13,0.13}{##1}}}
\expandafter\def\csname PY@tok@si\endcsname{\let\PY@bf=\textbf\def\PY@tc##1{\textcolor[rgb]{0.73,0.40,0.53}{##1}}}
\expandafter\def\csname PY@tok@se\endcsname{\let\PY@bf=\textbf\def\PY@tc##1{\textcolor[rgb]{0.73,0.40,0.13}{##1}}}
\expandafter\def\csname PY@tok@sr\endcsname{\def\PY@tc##1{\textcolor[rgb]{0.73,0.40,0.53}{##1}}}
\expandafter\def\csname PY@tok@ss\endcsname{\def\PY@tc##1{\textcolor[rgb]{0.10,0.09,0.49}{##1}}}
\expandafter\def\csname PY@tok@sx\endcsname{\def\PY@tc##1{\textcolor[rgb]{0.00,0.50,0.00}{##1}}}
\expandafter\def\csname PY@tok@m\endcsname{\def\PY@tc##1{\textcolor[rgb]{0.40,0.40,0.40}{##1}}}
\expandafter\def\csname PY@tok@gh\endcsname{\let\PY@bf=\textbf\def\PY@tc##1{\textcolor[rgb]{0.00,0.00,0.50}{##1}}}
\expandafter\def\csname PY@tok@gu\endcsname{\let\PY@bf=\textbf\def\PY@tc##1{\textcolor[rgb]{0.50,0.00,0.50}{##1}}}
\expandafter\def\csname PY@tok@gd\endcsname{\def\PY@tc##1{\textcolor[rgb]{0.63,0.00,0.00}{##1}}}
\expandafter\def\csname PY@tok@gi\endcsname{\def\PY@tc##1{\textcolor[rgb]{0.00,0.63,0.00}{##1}}}
\expandafter\def\csname PY@tok@gr\endcsname{\def\PY@tc##1{\textcolor[rgb]{1.00,0.00,0.00}{##1}}}
\expandafter\def\csname PY@tok@ge\endcsname{\let\PY@it=\textit}
\expandafter\def\csname PY@tok@gs\endcsname{\let\PY@bf=\textbf}
\expandafter\def\csname PY@tok@gp\endcsname{\let\PY@bf=\textbf\def\PY@tc##1{\textcolor[rgb]{0.00,0.00,0.50}{##1}}}
\expandafter\def\csname PY@tok@go\endcsname{\def\PY@tc##1{\textcolor[rgb]{0.53,0.53,0.53}{##1}}}
\expandafter\def\csname PY@tok@gt\endcsname{\def\PY@tc##1{\textcolor[rgb]{0.00,0.27,0.87}{##1}}}
\expandafter\def\csname PY@tok@err\endcsname{\def\PY@bc##1{\setlength{\fboxsep}{0pt}\fcolorbox[rgb]{1.00,0.00,0.00}{1,1,1}{\strut ##1}}}
\expandafter\def\csname PY@tok@kc\endcsname{\let\PY@bf=\textbf\def\PY@tc##1{\textcolor[rgb]{0.00,0.50,0.00}{##1}}}
\expandafter\def\csname PY@tok@kd\endcsname{\let\PY@bf=\textbf\def\PY@tc##1{\textcolor[rgb]{0.00,0.50,0.00}{##1}}}
\expandafter\def\csname PY@tok@kn\endcsname{\let\PY@bf=\textbf\def\PY@tc##1{\textcolor[rgb]{0.00,0.50,0.00}{##1}}}
\expandafter\def\csname PY@tok@kr\endcsname{\let\PY@bf=\textbf\def\PY@tc##1{\textcolor[rgb]{0.00,0.50,0.00}{##1}}}
\expandafter\def\csname PY@tok@bp\endcsname{\def\PY@tc##1{\textcolor[rgb]{0.00,0.50,0.00}{##1}}}
\expandafter\def\csname PY@tok@fm\endcsname{\def\PY@tc##1{\textcolor[rgb]{0.00,0.00,1.00}{##1}}}
\expandafter\def\csname PY@tok@vc\endcsname{\def\PY@tc##1{\textcolor[rgb]{0.10,0.09,0.49}{##1}}}
\expandafter\def\csname PY@tok@vg\endcsname{\def\PY@tc##1{\textcolor[rgb]{0.10,0.09,0.49}{##1}}}
\expandafter\def\csname PY@tok@vi\endcsname{\def\PY@tc##1{\textcolor[rgb]{0.10,0.09,0.49}{##1}}}
\expandafter\def\csname PY@tok@vm\endcsname{\def\PY@tc##1{\textcolor[rgb]{0.10,0.09,0.49}{##1}}}
\expandafter\def\csname PY@tok@sa\endcsname{\def\PY@tc##1{\textcolor[rgb]{0.73,0.13,0.13}{##1}}}
\expandafter\def\csname PY@tok@sb\endcsname{\def\PY@tc##1{\textcolor[rgb]{0.73,0.13,0.13}{##1}}}
\expandafter\def\csname PY@tok@sc\endcsname{\def\PY@tc##1{\textcolor[rgb]{0.73,0.13,0.13}{##1}}}
\expandafter\def\csname PY@tok@dl\endcsname{\def\PY@tc##1{\textcolor[rgb]{0.73,0.13,0.13}{##1}}}
\expandafter\def\csname PY@tok@s2\endcsname{\def\PY@tc##1{\textcolor[rgb]{0.73,0.13,0.13}{##1}}}
\expandafter\def\csname PY@tok@sh\endcsname{\def\PY@tc##1{\textcolor[rgb]{0.73,0.13,0.13}{##1}}}
\expandafter\def\csname PY@tok@s1\endcsname{\def\PY@tc##1{\textcolor[rgb]{0.73,0.13,0.13}{##1}}}
\expandafter\def\csname PY@tok@mb\endcsname{\def\PY@tc##1{\textcolor[rgb]{0.40,0.40,0.40}{##1}}}
\expandafter\def\csname PY@tok@mf\endcsname{\def\PY@tc##1{\textcolor[rgb]{0.40,0.40,0.40}{##1}}}
\expandafter\def\csname PY@tok@mh\endcsname{\def\PY@tc##1{\textcolor[rgb]{0.40,0.40,0.40}{##1}}}
\expandafter\def\csname PY@tok@mi\endcsname{\def\PY@tc##1{\textcolor[rgb]{0.40,0.40,0.40}{##1}}}
\expandafter\def\csname PY@tok@il\endcsname{\def\PY@tc##1{\textcolor[rgb]{0.40,0.40,0.40}{##1}}}
\expandafter\def\csname PY@tok@mo\endcsname{\def\PY@tc##1{\textcolor[rgb]{0.40,0.40,0.40}{##1}}}
\expandafter\def\csname PY@tok@ch\endcsname{\let\PY@it=\textit\def\PY@tc##1{\textcolor[rgb]{0.25,0.50,0.50}{##1}}}
\expandafter\def\csname PY@tok@cm\endcsname{\let\PY@it=\textit\def\PY@tc##1{\textcolor[rgb]{0.25,0.50,0.50}{##1}}}
\expandafter\def\csname PY@tok@cpf\endcsname{\let\PY@it=\textit\def\PY@tc##1{\textcolor[rgb]{0.25,0.50,0.50}{##1}}}
\expandafter\def\csname PY@tok@c1\endcsname{\let\PY@it=\textit\def\PY@tc##1{\textcolor[rgb]{0.25,0.50,0.50}{##1}}}
\expandafter\def\csname PY@tok@cs\endcsname{\let\PY@it=\textit\def\PY@tc##1{\textcolor[rgb]{0.25,0.50,0.50}{##1}}}

\def\PYZbs{\char`\\}
\def\PYZus{\char`\_}
\def\PYZob{\char`\{}
\def\PYZcb{\char`\}}
\def\PYZca{\char`\^}
\def\PYZam{\char`\&}
\def\PYZlt{\char`\<}
\def\PYZgt{\char`\>}
\def\PYZsh{\char`\#}
\def\PYZpc{\char`\%}
\def\PYZdl{\char`\$}
\def\PYZhy{\char`\-}
\def\PYZsq{\char`\'}
\def\PYZdq{\char`\"}
\def\PYZti{\char`\~}
% for compatibility with earlier versions
\def\PYZat{@}
\def\PYZlb{[}
\def\PYZrb{]}
\makeatother


    % Exact colors from NB
    \definecolor{incolor}{rgb}{0.0, 0.0, 0.5}
    \definecolor{outcolor}{rgb}{0.545, 0.0, 0.0}



    
    % Prevent overflowing lines due to hard-to-break entities
    \sloppy 
    % Setup hyperref package
    \hypersetup{
      breaklinks=true,  % so long urls are correctly broken across lines
      colorlinks=true,
      urlcolor=urlcolor,
      linkcolor=linkcolor,
      citecolor=citecolor,
      }
    % Slightly bigger margins than the latex defaults
    
    \geometry{verbose,tmargin=1in,bmargin=1in,lmargin=1in,rmargin=1in}
    
    

    \begin{document}
    
    
    \maketitle
    
    

    
    \section{Homework 4}\label{homework-4}

Jonah Spicher

    \begin{Verbatim}[commandchars=\\\{\}]
{\color{incolor}In [{\color{incolor}88}]:} \PY{c+c1}{\PYZsh{} Configure Jupyter so figures appear in the notebook}
         \PY{o}{\PYZpc{}}\PY{k}{matplotlib} inline
         
         \PY{c+c1}{\PYZsh{} Configure Jupyter to display the assigned value after an assignment}
         \PY{o}{\PYZpc{}}\PY{k}{config} InteractiveShell.ast\PYZus{}node\PYZus{}interactivity=\PYZsq{}last\PYZus{}expr\PYZus{}or\PYZus{}assign\PYZsq{}
         
         \PY{c+c1}{\PYZsh{} import classes from thinkbayes2}
         \PY{k+kn}{from} \PY{n+nn}{thinkbayes2} \PY{k}{import} \PY{n}{Pmf}\PY{p}{,} \PY{n}{Suite}\PY{p}{,} \PY{n}{Joint}\PY{p}{,} \PY{n}{NormalPdf}\PY{p}{,} \PY{n}{MakeMixture}\PY{p}{,} \PY{n}{MakeJoint}
         
         \PY{k+kn}{import} \PY{n+nn}{thinkplot}
         \PY{k+kn}{import} \PY{n+nn}{math}
         \PY{k+kn}{import} \PY{n+nn}{numpy} \PY{k}{as} \PY{n+nn}{np}
         \PY{k+kn}{from} \PY{n+nn}{scipy}\PY{n+nn}{.}\PY{n+nn}{stats} \PY{k}{import} \PY{n}{norm}
\end{Verbatim}


    \subsubsection{Height Problem}\label{height-problem}

The solution code was a lot prettier than mine so I definitely borrowed
some of it. Sorry.

    \begin{Verbatim}[commandchars=\\\{\}]
{\color{incolor}In [{\color{incolor}8}]:} \PY{n}{dist\PYZus{}height} \PY{o}{=} \PY{n+nb}{dict}\PY{p}{(}\PY{n}{male}\PY{o}{=}\PY{n}{norm}\PY{p}{(}\PY{l+m+mi}{178}\PY{p}{,} \PY{l+m+mf}{7.7}\PY{p}{)}\PY{p}{,}
                           \PY{n}{female}\PY{o}{=}\PY{n}{norm}\PY{p}{(}\PY{l+m+mi}{163}\PY{p}{,} \PY{l+m+mf}{7.3}\PY{p}{)}\PY{p}{)}
        \PY{n}{h\PYZus{}distribution} \PY{o}{=} \PY{n+nb}{dict}\PY{p}{(}\PY{n}{male}\PY{o}{=}\PY{n}{norm}\PY{p}{(}\PY{l+m+mi}{178}\PY{p}{,} \PY{l+m+mf}{7.7}\PY{p}{)}\PY{p}{,}
                           \PY{n}{female}\PY{o}{=}\PY{n}{norm}\PY{p}{(}\PY{l+m+mi}{163}\PY{p}{,} \PY{l+m+mf}{7.3}\PY{p}{)}\PY{p}{)}
        \PY{n}{hs} \PY{o}{=} \PY{n}{np}\PY{o}{.}\PY{n}{linspace}\PY{p}{(}\PY{l+m+mi}{130}\PY{p}{,} \PY{l+m+mi}{210}\PY{p}{)}
        \PY{n}{ps} \PY{o}{=} \PY{n}{dist\PYZus{}height}\PY{p}{[}\PY{l+s+s1}{\PYZsq{}}\PY{l+s+s1}{male}\PY{l+s+s1}{\PYZsq{}}\PY{p}{]}\PY{o}{.}\PY{n}{pdf}\PY{p}{(}\PY{n}{hs}\PY{p}{)}
        \PY{n}{male\PYZus{}height\PYZus{}pmf} \PY{o}{=} \PY{n}{Pmf}\PY{p}{(}\PY{n+nb}{dict}\PY{p}{(}\PY{n+nb}{zip}\PY{p}{(}\PY{n}{hs}\PY{p}{,} \PY{n}{ps}\PY{p}{)}\PY{p}{)}\PY{p}{)}\PY{p}{;}
        \PY{n}{ps} \PY{o}{=} \PY{n}{dist\PYZus{}height}\PY{p}{[}\PY{l+s+s1}{\PYZsq{}}\PY{l+s+s1}{female}\PY{l+s+s1}{\PYZsq{}}\PY{p}{]}\PY{o}{.}\PY{n}{pdf}\PY{p}{(}\PY{n}{hs}\PY{p}{)}
        \PY{n}{female\PYZus{}height\PYZus{}pmf} \PY{o}{=} \PY{n}{Pmf}\PY{p}{(}\PY{n+nb}{dict}\PY{p}{(}\PY{n+nb}{zip}\PY{p}{(}\PY{n}{hs}\PY{p}{,} \PY{n}{ps}\PY{p}{)}\PY{p}{)}\PY{p}{)}\PY{p}{;}
        \PY{n}{thinkplot}\PY{o}{.}\PY{n}{Pdf}\PY{p}{(}\PY{n}{male\PYZus{}height\PYZus{}pmf}\PY{p}{)}
        \PY{n}{thinkplot}\PY{o}{.}\PY{n}{Pdf}\PY{p}{(}\PY{n}{female\PYZus{}height\PYZus{}pmf}\PY{p}{)}
\end{Verbatim}


    \begin{center}
    \adjustimage{max size={0.9\linewidth}{0.9\paperheight}}{output_3_0.png}
    \end{center}
    { \hspace*{\fill} \\}
    
    \begin{Verbatim}[commandchars=\\\{\}]
{\color{incolor}In [{\color{incolor}9}]:} \PY{n}{metapmf} \PY{o}{=} \PY{n}{Pmf}\PY{p}{(}\PY{p}{\PYZob{}}\PY{n}{male\PYZus{}height\PYZus{}pmf}\PY{p}{:}\PY{l+m+mf}{0.49}\PY{p}{,} \PY{n}{female\PYZus{}height\PYZus{}pmf}\PY{p}{:}\PY{l+m+mf}{0.51}\PY{p}{\PYZcb{}}\PY{p}{)}
        \PY{n}{mix} \PY{o}{=} \PY{n}{MakeMixture}\PY{p}{(}\PY{n}{metapmf}\PY{p}{)}
        \PY{n}{thinkplot}\PY{o}{.}\PY{n}{Pdf}\PY{p}{(}\PY{n}{mix}\PY{p}{)}
\end{Verbatim}


    \begin{center}
    \adjustimage{max size={0.9\linewidth}{0.9\paperheight}}{output_4_0.png}
    \end{center}
    { \hspace*{\fill} \\}
    
    \begin{Verbatim}[commandchars=\\\{\}]
{\color{incolor}In [{\color{incolor}45}]:} \PY{k}{class} \PY{n+nc}{Heights}\PY{p}{(}\PY{n}{Suite}\PY{p}{,} \PY{n}{Joint}\PY{p}{)}\PY{p}{:}
             
             \PY{k}{def} \PY{n+nf}{Likelihood}\PY{p}{(}\PY{n+nb+bp}{self}\PY{p}{,} \PY{n}{data}\PY{p}{,} \PY{n}{hypo}\PY{p}{)}\PY{p}{:}
                 \PY{l+s+sd}{\PYZdq{}\PYZdq{}\PYZdq{}}
         \PY{l+s+sd}{        }
         \PY{l+s+sd}{        data: \PYZsq{}A\PYZsq{} or \PYZsq{}B\PYZsq{} is taller}
         \PY{l+s+sd}{        hypo: height of 1, }
         \PY{l+s+sd}{        \PYZdq{}\PYZdq{}\PYZdq{}}
                 \PY{n}{h1}\PY{p}{,} \PY{n}{h2} \PY{o}{=} \PY{n}{hypo}
                 \PY{k}{if} \PY{n}{data} \PY{o}{==} \PY{l+s+s1}{\PYZsq{}}\PY{l+s+s1}{A}\PY{l+s+s1}{\PYZsq{}}\PY{p}{:}
                     \PY{k}{return} \PY{l+m+mi}{1} \PY{k}{if} \PY{n}{h1} \PY{o}{\PYZgt{}} \PY{n}{h2} \PY{k}{else} \PY{l+m+mi}{0}
                 \PY{k}{else}\PY{p}{:}
                     \PY{k}{return} \PY{l+m+mi}{1} \PY{k}{if} \PY{n}{h2} \PY{o}{\PYZgt{}} \PY{n}{h1} \PY{k}{else} \PY{l+m+mi}{0}
\end{Verbatim}


    \begin{Verbatim}[commandchars=\\\{\}]
{\color{incolor}In [{\color{incolor}5}]:} \PY{k}{def} \PY{n+nf}{make\PYZus{}prior}\PY{p}{(}\PY{n}{A}\PY{p}{,} \PY{n}{B}\PY{p}{)}\PY{p}{:}
            \PY{n}{suite} \PY{o}{=} \PY{n}{Heights}\PY{p}{(}\PY{p}{)}
        
            \PY{k}{for} \PY{n}{h1}\PY{p}{,} \PY{n}{p1} \PY{o+ow}{in} \PY{n}{A}\PY{o}{.}\PY{n}{Items}\PY{p}{(}\PY{p}{)}\PY{p}{:}
                \PY{k}{for} \PY{n}{h2}\PY{p}{,} \PY{n}{p2} \PY{o+ow}{in} \PY{n}{B}\PY{o}{.}\PY{n}{Items}\PY{p}{(}\PY{p}{)}\PY{p}{:}
                    \PY{n}{suite}\PY{p}{[}\PY{n}{h1}\PY{p}{,} \PY{n}{h2}\PY{p}{]} \PY{o}{=} \PY{n}{p1} \PY{o}{*} \PY{n}{p2}
            \PY{k}{return} \PY{n}{suite}
        
        \PY{k}{def} \PY{n+nf}{faceoff}\PY{p}{(}\PY{n}{player1}\PY{p}{,} \PY{n}{player2}\PY{p}{,} \PY{n}{data}\PY{p}{)}\PY{p}{:}
            \PY{l+s+sd}{\PYZdq{}\PYZdq{}\PYZdq{}Compute the posterior distributions for both players.}
        \PY{l+s+sd}{    }
        \PY{l+s+sd}{    player1: Pmf}
        \PY{l+s+sd}{    player2: Pmf}
        \PY{l+s+sd}{    data: margin by which player1 beats player2}
        \PY{l+s+sd}{    \PYZdq{}\PYZdq{}\PYZdq{}}
            \PY{n}{joint} \PY{o}{=} \PY{n}{make\PYZus{}prior}\PY{p}{(}\PY{n}{player1}\PY{p}{,} \PY{n}{player2}\PY{p}{)}
            \PY{n}{joint}\PY{o}{.}\PY{n}{Update}\PY{p}{(}\PY{n}{data}\PY{p}{)}
            \PY{k}{return} \PY{n}{joint}\PY{o}{.}\PY{n}{Marginal}\PY{p}{(}\PY{l+m+mi}{0}\PY{p}{)}\PY{p}{,} \PY{n}{joint}\PY{o}{.}\PY{n}{Marginal}\PY{p}{(}\PY{l+m+mi}{1}\PY{p}{)}
\end{Verbatim}


    \begin{Verbatim}[commandchars=\\\{\}]
{\color{incolor}In [{\color{incolor}6}]:} \PY{n}{A} \PY{o}{=} \PY{n}{mix}
        \PY{n}{B} \PY{o}{=} \PY{n}{mix}
        \PY{k}{for} \PY{n}{i} \PY{o+ow}{in} \PY{n+nb}{range}\PY{p}{(}\PY{l+m+mi}{8}\PY{p}{)}\PY{p}{:}
            \PY{n}{A}\PY{p}{,} \PY{n}{\PYZus{}} \PY{o}{=} \PY{n}{faceoff}\PY{p}{(}\PY{n}{A}\PY{p}{,} \PY{n}{B}\PY{p}{,} \PY{l+s+s1}{\PYZsq{}}\PY{l+s+s1}{A}\PY{l+s+s1}{\PYZsq{}}\PY{p}{)}
            
        \PY{n}{A}\PY{p}{,} \PY{n}{B} \PY{o}{=} \PY{n}{faceoff}\PY{p}{(}\PY{n}{A}\PY{p}{,} \PY{n}{B}\PY{p}{,} \PY{l+s+s1}{\PYZsq{}}\PY{l+s+s1}{B}\PY{l+s+s1}{\PYZsq{}}\PY{p}{)}\PY{p}{;}
        
        \PY{n}{thinkplot}\PY{o}{.}\PY{n}{Pdf}\PY{p}{(}\PY{n}{A}\PY{p}{)}
        \PY{n}{A}\PY{o}{.}\PY{n}{Mean}\PY{p}{(}\PY{p}{)}
\end{Verbatim}


\begin{Verbatim}[commandchars=\\\{\}]
{\color{outcolor}Out[{\color{outcolor}6}]:} 181.60660153115964
\end{Verbatim}
            
    \begin{center}
    \adjustimage{max size={0.9\linewidth}{0.9\paperheight}}{output_7_1.png}
    \end{center}
    { \hspace*{\fill} \\}
    
    \subsubsection{Lincoln Index Problem}\label{lincoln-index-problem}

My solution seems to agree with the lincoln index pretty well, though I
am concerned that it might not take into account the possibility that
one tester is better than the other.

    \begin{Verbatim}[commandchars=\\\{\}]
{\color{incolor}In [{\color{incolor}50}]:} \PY{k+kn}{from} \PY{n+nn}{scipy}\PY{n+nn}{.}\PY{n+nn}{special} \PY{k}{import} \PY{n}{binom}
         
         \PY{k}{class} \PY{n+nc}{Bugs}\PY{p}{(}\PY{n}{Suite}\PY{p}{)}\PY{p}{:}
             \PY{k}{def} \PY{n+nf}{Likelihood}\PY{p}{(}\PY{n+nb+bp}{self}\PY{p}{,} \PY{n}{data}\PY{p}{,} \PY{n}{hypo}\PY{p}{)}\PY{p}{:}
                 \PY{l+s+sd}{\PYZdq{}\PYZdq{}\PYZdq{}Computes the likelihood of the data under the hypothesis.}
         
         \PY{l+s+sd}{        hypo: total bugs (N)}
         \PY{l+s+sd}{        data: \PYZsh{} caught by first tester (K), \PYZsh{} caught by second tester (n), \PYZsh{} caught by both (k)}
         \PY{l+s+sd}{        \PYZdq{}\PYZdq{}\PYZdq{}}
                 \PY{n}{N} \PY{o}{=} \PY{n}{hypo}
                 \PY{n}{K}\PY{p}{,} \PY{n}{n}\PY{p}{,} \PY{n}{k} \PY{o}{=} \PY{n}{data}
         
                 \PY{k}{if} \PY{n}{hypo} \PY{o}{\PYZlt{}} \PY{n}{K} \PY{o}{+} \PY{p}{(}\PY{n}{n} \PY{o}{\PYZhy{}} \PY{n}{k}\PY{p}{)}\PY{p}{:}
                     \PY{k}{return} \PY{l+m+mi}{0}
         
                 \PY{n}{like} \PY{o}{=} \PY{n}{binom}\PY{p}{(}\PY{n}{N}\PY{o}{\PYZhy{}}\PY{n}{K}\PY{p}{,} \PY{n}{n}\PY{o}{\PYZhy{}}\PY{n}{k}\PY{p}{)} \PY{o}{/} \PY{n}{binom}\PY{p}{(}\PY{n}{N}\PY{p}{,} \PY{n}{n}\PY{p}{)}
                 \PY{k}{return} \PY{n}{like}
\end{Verbatim}


    \begin{Verbatim}[commandchars=\\\{\}]
{\color{incolor}In [{\color{incolor}55}]:} \PY{n}{hypos} \PY{o}{=} \PY{n+nb}{range}\PY{p}{(}\PY{l+m+mi}{0}\PY{p}{,} \PY{l+m+mi}{201}\PY{p}{)}
         \PY{n}{suite} \PY{o}{=} \PY{n}{Bugs}\PY{p}{(}\PY{n}{hypos}\PY{p}{)}
         
         \PY{n}{data} \PY{o}{=} \PY{l+m+mi}{20}\PY{p}{,} \PY{l+m+mi}{1}\PY{p}{,} \PY{l+m+mi}{1}
         \PY{n}{suite}\PY{o}{.}\PY{n}{Update}\PY{p}{(}\PY{n}{data}\PY{p}{)}
         \PY{n}{thinkplot}\PY{o}{.}\PY{n}{plot}\PY{p}{(}\PY{n}{suite}\PY{p}{)}
         \PY{n}{suite}\PY{o}{.}\PY{n}{MAP}\PY{p}{(}\PY{p}{)}
\end{Verbatim}


\begin{Verbatim}[commandchars=\\\{\}]
{\color{outcolor}Out[{\color{outcolor}55}]:} 20
\end{Verbatim}
            
    \begin{center}
    \adjustimage{max size={0.9\linewidth}{0.9\paperheight}}{output_10_1.png}
    \end{center}
    { \hspace*{\fill} \\}
    
    \begin{Verbatim}[commandchars=\\\{\}]
{\color{incolor}In [{\color{incolor}56}]:} \PY{k}{def} \PY{n+nf}{lincoln\PYZus{}index}\PY{p}{(}\PY{n}{n1}\PY{p}{,} \PY{n}{n2}\PY{p}{,} \PY{n}{c}\PY{p}{)}\PY{p}{:}
             \PY{k}{return} \PY{n}{n1}\PY{o}{*}\PY{n}{n2}\PY{o}{/}\PY{n}{c}
\end{Verbatim}


    \begin{Verbatim}[commandchars=\\\{\}]
{\color{incolor}In [{\color{incolor}57}]:} \PY{n}{lincoln\PYZus{}index}\PY{p}{(}\PY{l+m+mi}{20}\PY{p}{,} \PY{l+m+mi}{1}\PY{p}{,} \PY{l+m+mi}{1}\PY{p}{)}
\end{Verbatim}


\begin{Verbatim}[commandchars=\\\{\}]
{\color{outcolor}Out[{\color{outcolor}57}]:} 20.0
\end{Verbatim}
            
    \subsubsection{Skeet Problem}\label{skeet-problem}

It should be fair to represent each shooter as a beta distribution,
where heads and tails are hits and misses.

    \begin{Verbatim}[commandchars=\\\{\}]
{\color{incolor}In [{\color{incolor}14}]:} \PY{k+kn}{from} \PY{n+nn}{thinkbayes2} \PY{k}{import} \PY{n}{Beta}
         \PY{n}{rhode} \PY{o}{=} \PY{n}{Beta}\PY{p}{(}\PY{l+m+mi}{1}\PY{p}{,} \PY{l+m+mi}{1}\PY{p}{)}
         \PY{n}{meng} \PY{o}{=} \PY{n}{Beta}\PY{p}{(}\PY{l+m+mi}{1}\PY{p}{,} \PY{l+m+mi}{1}\PY{p}{)}
         \PY{n}{r\PYZus{}hits} \PY{o}{=} \PY{l+m+mi}{15}\PY{o}{+}\PY{l+m+mi}{1}\PY{o}{+}\PY{l+m+mi}{2}\PY{o}{+}\PY{l+m+mi}{2}\PY{o}{+}\PY{l+m+mi}{2}
         \PY{n}{r\PYZus{}misses} \PY{o}{=} \PY{l+m+mi}{10}\PY{o}{+}\PY{l+m+mi}{1}
         \PY{n}{m\PYZus{}hits} \PY{o}{=} \PY{l+m+mi}{15}\PY{o}{+}\PY{l+m+mi}{1}\PY{o}{+}\PY{l+m+mi}{2}\PY{o}{+}\PY{l+m+mi}{2}\PY{o}{+}\PY{l+m+mi}{1}
         \PY{n}{m\PYZus{}misses} \PY{o}{=} \PY{l+m+mi}{10}\PY{o}{+}\PY{l+m+mi}{1}\PY{o}{+}\PY{l+m+mi}{1}
         \PY{n}{rhode}\PY{o}{.}\PY{n}{Update}\PY{p}{(}\PY{p}{(}\PY{n}{r\PYZus{}hits}\PY{p}{,} \PY{n}{r\PYZus{}misses}\PY{p}{)}\PY{p}{)}
         \PY{n}{meng}\PY{o}{.}\PY{n}{Update}\PY{p}{(}\PY{p}{(}\PY{n}{m\PYZus{}hits}\PY{p}{,} \PY{n}{m\PYZus{}misses}\PY{p}{)}\PY{p}{)}
         \PY{n}{rhode} \PY{o}{=} \PY{n}{rhode}\PY{o}{.}\PY{n}{MakePmf}\PY{p}{(}\PY{p}{)}
         \PY{n}{meng} \PY{o}{=} \PY{n}{meng}\PY{o}{.}\PY{n}{MakePmf}\PY{p}{(}\PY{p}{)}
         \PY{n}{thinkplot}\PY{o}{.}\PY{n}{plot}\PY{p}{(}\PY{n}{rhode}\PY{p}{,} \PY{n}{color}\PY{o}{=}\PY{l+s+s1}{\PYZsq{}}\PY{l+s+s1}{grey}\PY{l+s+s1}{\PYZsq{}}\PY{p}{)}
         \PY{n}{thinkplot}\PY{o}{.}\PY{n}{plot}\PY{p}{(}\PY{n}{meng}\PY{p}{)}
\end{Verbatim}


    \begin{center}
    \adjustimage{max size={0.9\linewidth}{0.9\paperheight}}{output_14_0.png}
    \end{center}
    { \hspace*{\fill} \\}
    
    Then, we can pit the shooters against eachother in several thousand
matches, until they converge on a defintie overall winner.

    \begin{Verbatim}[commandchars=\\\{\}]
{\color{incolor}In [{\color{incolor}15}]:} \PY{k+kn}{import} \PY{n+nn}{random}
         \PY{k}{def} \PY{n+nf}{flip}\PY{p}{(}\PY{n}{p}\PY{p}{)}\PY{p}{:}
             \PY{k}{return} \PY{k+kc}{True} \PY{k}{if} \PY{n}{random}\PY{o}{.}\PY{n}{random}\PY{p}{(}\PY{p}{)} \PY{o}{\PYZlt{}} \PY{n}{p}  \PY{k}{else} \PY{k+kc}{False}
         
         \PY{k}{def} \PY{n+nf}{rematch}\PY{p}{(}\PY{n}{r}\PY{p}{,} \PY{n}{m}\PY{p}{,} \PY{n}{num\PYZus{}matches}\PY{p}{)}\PY{p}{:}
             \PY{n}{outcomes} \PY{o}{=} \PY{p}{[}\PY{l+m+mi}{0}\PY{p}{,} \PY{l+m+mi}{0}\PY{p}{,} \PY{l+m+mi}{0}\PY{p}{]} \PY{c+c1}{\PYZsh{} rwins, mwins, ties}
             \PY{k}{for} \PY{n}{i} \PY{o+ow}{in} \PY{n+nb}{range}\PY{p}{(}\PY{n}{num\PYZus{}matches}\PY{p}{)}\PY{p}{:}
                 \PY{n}{r\PYZus{}total} \PY{o}{=} \PY{l+m+mi}{0}
                 \PY{n}{m\PYZus{}total} \PY{o}{=} \PY{l+m+mi}{0}
                 
                 \PY{n}{r\PYZus{}ps} \PY{o}{=} \PY{n}{r}\PY{o}{.}\PY{n}{Sample}\PY{p}{(}\PY{l+m+mi}{25}\PY{p}{)}
                 \PY{n}{m\PYZus{}ps} \PY{o}{=} \PY{n}{m}\PY{o}{.}\PY{n}{Sample}\PY{p}{(}\PY{l+m+mi}{25}\PY{p}{)}
                 
                 \PY{k}{for} \PY{n}{i} \PY{o+ow}{in} \PY{n+nb}{range}\PY{p}{(}\PY{l+m+mi}{25}\PY{p}{)}\PY{p}{:}
                     
                     \PY{k}{if} \PY{n}{flip}\PY{p}{(}\PY{n}{r\PYZus{}ps}\PY{p}{[}\PY{n}{i}\PY{p}{]}\PY{p}{)}\PY{p}{:}
                         \PY{n}{r\PYZus{}total} \PY{o}{+}\PY{o}{=} \PY{l+m+mi}{1}
                     \PY{k}{if} \PY{n}{flip}\PY{p}{(}\PY{n}{m\PYZus{}ps}\PY{p}{[}\PY{n}{i}\PY{p}{]}\PY{p}{)}\PY{p}{:}
                         \PY{n}{m\PYZus{}total} \PY{o}{+}\PY{o}{=} \PY{l+m+mi}{1}
                 \PY{k}{if} \PY{n}{r\PYZus{}total} \PY{o}{\PYZgt{}} \PY{n}{m\PYZus{}total}\PY{p}{:}
                     \PY{n}{outcomes}\PY{p}{[}\PY{l+m+mi}{0}\PY{p}{]} \PY{o}{+}\PY{o}{=} \PY{l+m+mi}{1}
                 \PY{k}{elif} \PY{n}{r\PYZus{}total} \PY{o}{\PYZlt{}} \PY{n}{m\PYZus{}total}\PY{p}{:}
                     \PY{n}{outcomes}\PY{p}{[}\PY{l+m+mi}{1}\PY{p}{]} \PY{o}{+}\PY{o}{=} \PY{l+m+mi}{1}
                 \PY{k}{else}\PY{p}{:}
                     \PY{n}{outcomes}\PY{p}{[}\PY{l+m+mi}{2}\PY{p}{]} \PY{o}{+}\PY{o}{=} \PY{l+m+mi}{1}
             \PY{k}{return} \PY{n}{outcomes}
                 
\end{Verbatim}


    \begin{Verbatim}[commandchars=\\\{\}]
{\color{incolor}In [{\color{incolor}16}]:} \PY{n}{num\PYZus{}matches} \PY{o}{=} \PY{l+m+mi}{100000}
         \PY{n}{outcomes} \PY{o}{=} \PY{n}{rematch}\PY{p}{(}\PY{n}{rhode}\PY{p}{,} \PY{n}{meng}\PY{p}{,} \PY{n}{num\PYZus{}matches}\PY{p}{)}
\end{Verbatim}


\begin{Verbatim}[commandchars=\\\{\}]
{\color{outcolor}Out[{\color{outcolor}16}]:} [52695, 35831, 11474]
\end{Verbatim}
            
    \begin{Verbatim}[commandchars=\\\{\}]
{\color{incolor}In [{\color{incolor}17}]:} \PY{n}{r\PYZus{}winrate} \PY{o}{=} \PY{n}{outcomes}\PY{p}{[}\PY{l+m+mi}{0}\PY{p}{]} \PY{o}{/} \PY{n}{num\PYZus{}matches}
         \PY{n}{m\PYZus{}winrate} \PY{o}{=} \PY{n}{outcomes}\PY{p}{[}\PY{l+m+mi}{1}\PY{p}{]} \PY{o}{/} \PY{n}{num\PYZus{}matches}
         \PY{n}{tie\PYZus{}rate} \PY{o}{=} \PY{n}{outcomes}\PY{p}{[}\PY{l+m+mi}{2}\PY{p}{]} \PY{o}{/} \PY{n}{num\PYZus{}matches}
         \PY{n+nb}{print}\PY{p}{(}\PY{n}{r\PYZus{}winrate}\PY{p}{,} \PY{n}{m\PYZus{}winrate}\PY{p}{)}
\end{Verbatim}


    \begin{Verbatim}[commandchars=\\\{\}]
0.52695 0.35831

    \end{Verbatim}

    It looks like in the longterm, rhode will win more games than meng,
though that is largely because a large portion of their games end in
tied. Rhode only wins a little more than half of the time.

    \subsubsection{Social Desirability
Problem}\label{social-desirability-problem}

Here, I started with a distribution of the number of people who flipped
heads. Then, I found the likelihood as the total probability of the data
given the hypothesis as a weighted sum for each possible number of
heads. This required assuming a prior prevalence of atheists, but
considering this was in a group of 100 people, that seems fair. The
number seems a little low, though, as a quick guess says that because
approximately half of the group should flip heads, the fraction of the
remainder that answers no should be representative of the whole.

    \begin{Verbatim}[commandchars=\\\{\}]
{\color{incolor}In [{\color{incolor}48}]:} \PY{k+kn}{from} \PY{n+nn}{thinkbayes2} \PY{k}{import} \PY{n}{MakeBinomialPmf}
         \PY{n}{heads\PYZus{}dist} \PY{o}{=} \PY{n}{MakeBinomialPmf}\PY{p}{(}\PY{l+m+mi}{100}\PY{p}{,} \PY{l+m+mf}{0.5}\PY{p}{)}
         \PY{n}{thinkplot}\PY{o}{.}\PY{n}{plot}\PY{p}{(}\PY{n}{heads\PYZus{}dist}\PY{p}{)}
\end{Verbatim}


    \begin{center}
    \adjustimage{max size={0.9\linewidth}{0.9\paperheight}}{output_21_0.png}
    \end{center}
    { \hspace*{\fill} \\}
    
    \begin{Verbatim}[commandchars=\\\{\}]
{\color{incolor}In [{\color{incolor}47}]:} \PY{n}{p\PYZus{}atheist} \PY{o}{=} \PY{l+m+mf}{0.2} \PY{c+c1}{\PYZsh{} A guess about the prevalence of atheists overall. Wouldn\PYZsq{}t work if we were actually estimating}
                         \PY{c+c1}{\PYZsh{} the frequency of atheists overall, but here it should be roughly true.}
         \PY{k}{class} \PY{n+nc}{Survey}\PY{p}{(}\PY{n}{Suite}\PY{p}{)}\PY{p}{:}
             \PY{k}{def} \PY{n+nf}{Likelihood}\PY{p}{(}\PY{n+nb+bp}{self}\PY{p}{,} \PY{n}{data}\PY{p}{,} \PY{n}{hypo}\PY{p}{)}\PY{p}{:}
                 \PY{l+s+sd}{\PYZdq{}\PYZdq{}\PYZdq{}}
         \PY{l+s+sd}{        data: number of people who answered \PYZsq{}yes\PYZsq{}}
         \PY{l+s+sd}{        hypo: number of people we think are atheists}
         \PY{l+s+sd}{        \PYZdq{}\PYZdq{}\PYZdq{}}
                 \PY{n}{total} \PY{o}{=} \PY{l+m+mi}{0}
                 \PY{k}{for} \PY{n}{num}\PY{p}{,} \PY{n}{p\PYZus{}heads} \PY{o+ow}{in} \PY{n}{heads\PYZus{}dist}\PY{o}{.}\PY{n}{Items}\PY{p}{(}\PY{p}{)}\PY{p}{:} \PY{c+c1}{\PYZsh{} Iterate through the possible numbers of heads and how likely that is    }
                     \PY{k}{if} \PY{n}{data} \PY{o}{\PYZlt{}} \PY{n}{num}\PY{p}{:} \PY{c+c1}{\PYZsh{} First, check to see if this is possible. Everyone who flipped heads had to say yes.}
                         \PY{n}{p\PYZus{}data} \PY{o}{=} \PY{l+m+mi}{0}
                     \PY{k}{else}\PY{p}{:}
                         \PY{n}{num\PYZus{}known\PYZus{}atheists} \PY{o}{=} \PY{l+m+mi}{100} \PY{o}{\PYZhy{}} \PY{n}{data} \PY{c+c1}{\PYZsh{} Anyone who didn\PYZsq{}t say yes is definitely an atheist}
                         \PY{n}{predicted\PYZus{}atheists} \PY{o}{=} \PY{n}{hypo} \PY{o}{\PYZhy{}} \PY{n}{num\PYZus{}known\PYZus{}atheists} \PY{c+c1}{\PYZsh{} Our hypothesis is that there is an extra amount of atheists in the group that }
                         \PY{k}{if} \PY{n}{hypo} \PY{o}{\PYZlt{}} \PY{n}{num\PYZus{}known\PYZus{}atheists}\PY{p}{:} \PY{c+c1}{\PYZsh{} If the data says there are more atheists than we think there are, then P(D|H) = 0}
                             \PY{n}{p\PYZus{}data} \PY{o}{=} \PY{l+m+mi}{0}
                         \PY{k}{else}\PY{p}{:}
                             \PY{n}{atheist\PYZus{}dist} \PY{o}{=} \PY{n}{MakeBinomialPmf}\PY{p}{(}\PY{n}{num}\PY{p}{,} \PY{n}{p\PYZus{}atheist}\PY{p}{)} \PY{c+c1}{\PYZsh{} This is a distribution for how many of the people who flipped coins are atheists}
                             \PY{n}{p\PYZus{}data} \PY{o}{=} \PY{n}{atheist\PYZus{}dist}\PY{p}{[}\PY{n}{predicted\PYZus{}atheists}\PY{p}{]} \PY{c+c1}{\PYZsh{} This is that distribution, evaluated at the predicted number of atheists}
                         
                     \PY{n}{total} \PY{o}{+}\PY{o}{=} \PY{n}{p\PYZus{}heads} \PY{o}{*} \PY{n}{p\PYZus{}data}
                 \PY{k}{return} \PY{n}{total}
\end{Verbatim}


    \begin{Verbatim}[commandchars=\\\{\}]
{\color{incolor}In [{\color{incolor}45}]:} \PY{n}{people} \PY{o}{=} \PY{n}{Survey}\PY{p}{(}\PY{n+nb}{range}\PY{p}{(}\PY{l+m+mi}{0}\PY{p}{,} \PY{l+m+mi}{101}\PY{p}{)}\PY{p}{)}
         \PY{n}{people}\PY{o}{.}\PY{n}{Update}\PY{p}{(}\PY{l+m+mi}{80}\PY{p}{)}
         \PY{n}{thinkplot}\PY{o}{.}\PY{n}{plot}\PY{p}{(}\PY{n}{people}\PY{p}{)}
         \PY{n}{people}\PY{o}{.}\PY{n}{Mean}\PY{p}{(}\PY{p}{)}
\end{Verbatim}


\begin{Verbatim}[commandchars=\\\{\}]
{\color{outcolor}Out[{\color{outcolor}45}]:} 29.99999999915436
\end{Verbatim}
            
    \begin{center}
    \adjustimage{max size={0.9\linewidth}{0.9\paperheight}}{output_23_1.png}
    \end{center}
    { \hspace*{\fill} \\}
    
    \subsubsection{Volunteer Problem}\label{volunteer-problem}

This definitely still seems wrong, and my solution is admittedly pretty
janky, but I am not sure how to fix it.

    \begin{Verbatim}[commandchars=\\\{\}]
{\color{incolor}In [{\color{incolor}94}]:} \PY{k}{class} \PY{n+nc}{first\PYZus{}update\PYZus{}volunteers}\PY{p}{(}\PY{n}{Suite}\PY{p}{,} \PY{n}{Joint}\PY{p}{)}\PY{p}{:}
             \PY{k}{def} \PY{n+nf}{Likelihood}\PY{p}{(}\PY{n+nb+bp}{self}\PY{p}{,} \PY{n}{data}\PY{p}{,} \PY{n}{hypo}\PY{p}{)}\PY{p}{:}
                 \PY{l+s+sd}{\PYZdq{}\PYZdq{}\PYZdq{}}
         \PY{l+s+sd}{        data: number reported back}
         \PY{l+s+sd}{        hypo: a tuple: (prob participating, prob reporting back)}
         \PY{l+s+sd}{        \PYZdq{}\PYZdq{}\PYZdq{}}
                 \PY{n}{q}\PY{p}{,} \PY{n}{r} \PY{o}{=} \PY{n}{hypo}
                 \PY{n}{report\PYZus{}dist} \PY{o}{=} \PY{n}{MakeBinomialPmf}\PY{p}{(}\PY{l+m+mi}{140}\PY{p}{,} \PY{n}{r}\PY{p}{)} \PY{c+c1}{\PYZsh{} Get a binomial distribution for current hypothetical r}
                 \PY{n}{partic\PYZus{}dist} \PY{o}{=} \PY{n}{MakeBinomialPmf}\PY{p}{(}\PY{l+m+mi}{140}\PY{p}{,} \PY{n}{q}\PY{p}{)} \PY{c+c1}{\PYZsh{} Get a binomial distribution for current hypothetical q}
                 \PY{n}{like} \PY{o}{=} \PY{n}{report\PYZus{}dist}\PY{p}{[}\PY{n}{data}\PY{p}{]} \PY{o}{*} \PY{n}{partic\PYZus{}dist}\PY{p}{[}\PY{n}{data}\PY{p}{]} \PY{c+c1}{\PYZsh{} Multiply them both called for the data}
                 
                 \PY{k}{return} \PY{n}{like}
\end{Verbatim}


    \begin{Verbatim}[commandchars=\\\{\}]
{\color{incolor}In [{\color{incolor}95}]:} \PY{n}{ps} \PY{o}{=} \PY{n}{np}\PY{o}{.}\PY{n}{linspace}\PY{p}{(}\PY{l+m+mi}{0}\PY{p}{,} \PY{l+m+mi}{1}\PY{p}{,} \PY{n}{num}\PY{o}{=}\PY{l+m+mi}{100}\PY{p}{)}
         \PY{n}{q\PYZus{}prior} \PY{o}{=} \PY{n}{Pmf}\PY{p}{(}\PY{n}{ps}\PY{p}{)}
         \PY{n}{r\PYZus{}prior} \PY{o}{=} \PY{n}{Pmf}\PY{p}{(}\PY{n}{ps}\PY{p}{)}
         \PY{n}{prior} \PY{o}{=} \PY{n}{MakeJoint}\PY{p}{(}\PY{n}{q\PYZus{}prior}\PY{p}{,} \PY{n}{r\PYZus{}prior}\PY{p}{)} \PY{c+c1}{\PYZsh{} Sets up an even prior}
         \PY{n}{prior} \PY{o}{=} \PY{n}{first\PYZus{}update\PYZus{}volunteers}\PY{p}{(}\PY{n}{prior}\PY{p}{)}
         \PY{n}{prior}\PY{o}{.}\PY{n}{Update}\PY{p}{(}\PY{l+m+mi}{50}\PY{p}{)} \PY{c+c1}{\PYZsh{} Updates given that we know fifty people reported back}
         \PY{n}{thinkplot}\PY{o}{.}\PY{n}{Contour}\PY{p}{(}\PY{n}{prior}\PY{p}{)}
\end{Verbatim}


    \begin{Verbatim}[commandchars=\\\{\}]
/home/jonah/anaconda3/lib/python3.6/site-packages/matplotlib/contour.py:960: UserWarning: The following kwargs were not used by contour: 'linewidth'
  s)

    \end{Verbatim}

    \begin{center}
    \adjustimage{max size={0.9\linewidth}{0.9\paperheight}}{output_26_1.png}
    \end{center}
    { \hspace*{\fill} \\}
    
    \begin{Verbatim}[commandchars=\\\{\}]
{\color{incolor}In [{\color{incolor}106}]:} \PY{n}{q} \PY{o}{=} \PY{n}{prior}\PY{o}{.}\PY{n}{Marginal}\PY{p}{(}\PY{l+m+mi}{0}\PY{p}{)}
          \PY{n}{r} \PY{o}{=} \PY{n}{prior}\PY{o}{.}\PY{n}{Marginal}\PY{p}{(}\PY{l+m+mi}{1}\PY{p}{)}
          \PY{n}{thinkplot}\PY{o}{.}\PY{n}{plot}\PY{p}{(}\PY{n}{q}\PY{p}{)}
          \PY{n}{thinkplot}\PY{o}{.}\PY{n}{plot}\PY{p}{(}\PY{n}{r}\PY{p}{)}
\end{Verbatim}


    \begin{center}
    \adjustimage{max size={0.9\linewidth}{0.9\paperheight}}{output_27_0.png}
    \end{center}
    { \hspace*{\fill} \\}
    
    \begin{Verbatim}[commandchars=\\\{\}]
{\color{incolor}In [{\color{incolor}96}]:} \PY{k}{class} \PY{n+nc}{second\PYZus{}update\PYZus{}volunteers}\PY{p}{(}\PY{n}{Suite}\PY{p}{,} \PY{n}{Joint}\PY{p}{)}\PY{p}{:}
             \PY{k}{def} \PY{n+nf}{Likelihood}\PY{p}{(}\PY{n+nb+bp}{self}\PY{p}{,} \PY{n}{data}\PY{p}{,} \PY{n}{hypo}\PY{p}{)}\PY{p}{:}
                 \PY{l+s+sd}{\PYZdq{}\PYZdq{}\PYZdq{}}
         \PY{l+s+sd}{        data: a tuple: (total, number participated, number reported back)}
         \PY{l+s+sd}{        hypo: a tuple: (prob participating, prob reporting back)}
         \PY{l+s+sd}{        \PYZdq{}\PYZdq{}\PYZdq{}}
                 \PY{n}{total}\PY{p}{,} \PY{n}{num\PYZus{}q}\PY{p}{,} \PY{n}{num\PYZus{}r} \PY{o}{=} \PY{n}{data}
                 \PY{n}{q}\PY{p}{,} \PY{n}{r}\PY{p}{,} \PY{o}{=} \PY{n}{hypo}
                 \PY{n}{report\PYZus{}dist} \PY{o}{=} \PY{n}{MakeBinomialPmf}\PY{p}{(}\PY{n}{total}\PY{p}{,} \PY{n}{r}\PY{p}{)} \PY{c+c1}{\PYZsh{} Makes a distribution for what we would expect to see for}
                 \PY{n}{partic\PYZus{}dist} \PY{o}{=} \PY{n}{MakeBinomialPmf}\PY{p}{(}\PY{n}{total}\PY{p}{,} \PY{n}{q}\PY{p}{)} \PY{c+c1}{\PYZsh{} five random people we call}
                 \PY{n}{p\PYZus{}q} \PY{o}{=} \PY{n}{partic\PYZus{}dist}\PY{p}{[}\PY{n}{num\PYZus{}q}\PY{p}{]} 
                 \PY{n}{p\PYZus{}r} \PY{o}{=} \PY{n}{partic\PYZus{}dist}\PY{p}{[}\PY{n}{num\PYZus{}r}\PY{p}{]}
                 \PY{n}{like} \PY{o}{=} \PY{n}{p\PYZus{}q} \PY{o}{*} \PY{n}{p\PYZus{}r} \PY{c+c1}{\PYZsh{} Just like the first likelihood function, updates given these probabilities}
                 \PY{k}{return} \PY{n}{like}
\end{Verbatim}


    \begin{Verbatim}[commandchars=\\\{\}]
{\color{incolor}In [{\color{incolor}99}]:} \PY{n}{post} \PY{o}{=} \PY{n}{second\PYZus{}update\PYZus{}volunteers}\PY{p}{(}\PY{n}{prior}\PY{p}{)}
         \PY{n}{post}\PY{o}{.}\PY{n}{Update}\PY{p}{(}\PY{p}{(}\PY{l+m+mi}{5}\PY{p}{,} \PY{l+m+mi}{3}\PY{p}{,} \PY{l+m+mi}{1}\PY{p}{)}\PY{p}{)}
         \PY{n}{thinkplot}\PY{o}{.}\PY{n}{Contour}\PY{p}{(}\PY{n}{post}\PY{p}{)}
\end{Verbatim}


    \begin{Verbatim}[commandchars=\\\{\}]
/home/jonah/anaconda3/lib/python3.6/site-packages/matplotlib/contour.py:960: UserWarning: The following kwargs were not used by contour: 'linewidth'
  s)

    \end{Verbatim}

    \begin{center}
    \adjustimage{max size={0.9\linewidth}{0.9\paperheight}}{output_29_1.png}
    \end{center}
    { \hspace*{\fill} \\}
    
    \begin{Verbatim}[commandchars=\\\{\}]
{\color{incolor}In [{\color{incolor}105}]:} \PY{n}{q} \PY{o}{=} \PY{n}{post}\PY{o}{.}\PY{n}{Marginal}\PY{p}{(}\PY{l+m+mi}{0}\PY{p}{)}
          \PY{n}{r} \PY{o}{=} \PY{n}{post}\PY{o}{.}\PY{n}{Marginal}\PY{p}{(}\PY{l+m+mi}{1}\PY{p}{)}
          \PY{n}{thinkplot}\PY{o}{.}\PY{n}{plot}\PY{p}{(}\PY{n}{q}\PY{p}{)}
          \PY{n}{thinkplot}\PY{o}{.}\PY{n}{plot}\PY{p}{(}\PY{n}{r}\PY{p}{)}
\end{Verbatim}


    \begin{center}
    \adjustimage{max size={0.9\linewidth}{0.9\paperheight}}{output_30_0.png}
    \end{center}
    { \hspace*{\fill} \\}
    
    \subsubsection{Project Ideas}\label{project-ideas}

\begin{enumerate}
\def\labelenumi{\arabic{enumi}.}
\item
  Civilization win rates based on starting factors, similar to what is
  done \href{https://civscience.wordpress.com/}{here}
\item
  Not sure if this counts as a pitch because you posted it, but
  estimating the number of atheists in the US looked like an intersting
  problem.
\item
  Something in linguistics, an example that's maybe silly is guessing
  what language an unknown word is based on the letters or phonemes it
  contains. Alternatively, something with Grimms law (and the equivalent
  for other languages) and finding the likelihood that different words
  come from Proto-Indo-European (this is based on sound shifts different
  languages underwent, some reading
  \href{https://en.wikipedia.org/wiki/Proto-Indo-European_language}{here}
  and \href{https://en.wikipedia.org/wiki/Comparative_method}{here} if
  you are curious). This is almost defintiely overscoped though, even if
  statistical methods can even be applied here.
\end{enumerate}


    % Add a bibliography block to the postdoc
    
    
    
    \end{document}
